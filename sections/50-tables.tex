\section{Тaблицы}

Простая таблица с номером~\ref{tab:simple}.

% \renewcommand{\arraystretch}{1.3}
\begin{table}[H]
\caption{Простая таблица}\label{tab:simple}
\begin{tabular}{|c|c|c|}
    \hline
    11 & 12 & 13\\\hline
    21 & 22 & 23\\\hline
    31 & 32 & 33\\\hline
\end{tabular}
\end{table}

\begin{table}[H]
\caption{Tabularx}\label{tab:x}
\begin{tabularx}{0.8\textwidth} { 
  | >{\raggedright\arraybackslash}X 
  | >{\centering\arraybackslash}X 
  | >{\raggedleft\arraybackslash}X | }
 \hline
     item 11 & item 12 & item 13\\\hline
     item 21  & item 22  & item 23\\\hline
\end{tabularx}
\end{table}

\begin{table}[H]
    \caption{Сложная таблица}\label{tab:hd}
\begin{tabular}{|*{10}{c|}}
    \hline
    \multirow{2}{*}{Параметр \(x_i\)} &
    \multicolumn{4}{c|}{Параметр \(x_j\)} &
    \multicolumn{2}{c|}{Первый шаг} &
    \multicolumn{2}{c|}{Второй шаг}\\\cline{2-9} &
    \(X_1\) & \(X_2\) & \(X_3\) & \(X_4\) & 
    \(w_i\) & \(K_{\text{в}i}\) &
    \(w_i\) & \(K_{\text{в}i}\)\\\hline
    \(X_1\) & 1 & 1 & 1.5 & 1.5 & 5 & 0.31 & 19 & 0.32\\\hline
    \(X_2\) & 1 & 1 & 1.5 & 1.5 & 5 & 0.31 & 19 & 0.32\\\hline
    \(X_3\) & 1 & 1 & 1.5 & 1.5 & 5 & 0.31 & 19 & 0.32\\\hline
    \(X_4\) & 1 & 1 & 1.5 & 1.5 & 5 & 0.31 & 19 & 0.32\\\hline
    \multicolumn{5}{|c|}{Итого:} & 16 & 1 & 59.5 & 1\\\hline
\end{tabular}
\end{table}

\begin{longtable}{|l|l|}
    \caption{Длинная таблица}\label{tab:long}\\\hline
    \multicolumn{1}{|c|}{\textbf{test}} & 
    \multicolumn{1}{c|}{\textbf{test2}} \\\hline 
\endfirsthead
    \caption*{Продолжение таблицы \ref{tab:long}}
\endhead
     Lots of lines & like this\\\hline
     Lots of lines & like this\\\hline
     Lots of lines & like this\\\hline
     Lots of lines & like this\\\hline
     Lots of lines & like this\\\hline
     Lots of lines & like this\\\hline
     Lots of lines & like this\\\hline
     Lots of lines & like this\\\hline
     Lots of lines & like this\\\hline
     Lots of lines & like this\\\hline
     Lots of lines & like this\\\hline
     Lots of lines & like this\\\hline
\end{longtable}

\begin{table}[H]
    \caption{Tabulary}\label{tab:y}
  \begin{tabulary}{\textwidth}{|L|L|L|}
  \hline
    Short sentences & \# & Long sentences \\\hline
    This is short. & 173 & This is much loooooooonger, because there are many more words. \\
    \hline
This is not shorter. & 317 & This is still loooooooonger, because there are many more words. \\
\hline
  \end{tabulary}  
\end{table}

% xltabular таблица с номером~\ref{tab:xltabular}
% \begin{xltabular}{\linewidth}{|m{0.15\linewidth}|m{0.15\linewidth}|m{0.15\linewidth}|m{0.15\linewidth}|}
%   \caption{xltabular}\label{tab:xltabular}\\\hline
%   Наименование поля & Тип данных &
%   Дополнительнае характеристика & Описание \\\hline
%   % \endfirsthead{}
%   % \caption*{Продолжение таблицы \ref{tab:xltabular}}
%   % \endhead{}
%   % \endfoot{}
%   % \endlastfoot{}
%   id & Int & Primary key & Идентификатор покупателя \\\hline
% \end{xltabular}
